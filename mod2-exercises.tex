\documentclass{article} 
\usepackage{amsmath}  
\usepackage{amsthm, amssymb}
\usepackage[a4paper,left=25mm,right=25mm,top=30mm,bottom=30mm]{geometry}
\usepackage{fancyhdr}
\usepackage{titlesec}
\usepackage{enumerate}
\usepackage{graphicx} 
\usepackage[dvipsnames]{xcolor}
\usepackage{transparent} 
\usepackage{parskip}
\usepackage{tikz} 
\usepackage{cancel}

\title{num theory exercises} 
\author{emilianna monami louise limlengco} 
\date{\today} 

\fboxsep=4pt
\renewcommand{\footnoterule}{\vfill\kern-3pt \hrule width 0.4\columnwidth\kern2.6pt} %yoinked from LSE
\renewcommand{\labelitemi}{$\rightarrow$}
\renewcommand{\labelenumi}{\colorbox{pink}{\textbf{\arabic{enumi}}}}
\renewcommand{\labelenumii}{\transparent{0.5}\colorbox{CornflowerBlue}{\transparent{1.0}\textbf{\alph{enumii}}}}

\newcommand{\multibinom}[2]{
  \left(\!\!\middle(\genfrac{}{}{0pt}{}{#1}{#2}\middle)\!\!\right)} %yoinked from LSE

\newenvironment{solution}
  {\renewcommand\qedsymbol{$\blacksquare$}\begin{proof}[Solution]}
  {\end{proof}}

\begin{document} 

\section*{How to Use this Reviewer}
Hello! This is a compilation of solved exercises for Module 1 of MATH 51.4. All of these exercises are taken straight from Aldrich and Cisco's course notes, so you can expect tests 
to be very similar to the items given. However, there are certain items that are much more difficult than what they expect us to do, and are mostly for nerds like me to geek out about 
on documents like these. I'll note when these items show up, so that you don't spend energy that you don't really need to trying to understand them.\par Normal items will look like this:\begin{enumerate} 
    \item A very easy math problem. What's 1 + 1?
\end{enumerate} 
whereas difficult problems will be soulless, like this:\begin{enumerate}\setcounter{enumi}{1}
    \renewcommand{\labelenumi}{\fcolorbox{magenta}{white}{\textbf{\arabic{enumi}}}}
    \item A very difficult math problem. Prove that $\displaystyle \binom{2n}{n} < 2^{2n-2},~\forall n \geq 5$ using induction. 
\end{enumerate} I might also include warnings in my \textbf{Nerd Interjections!}\par
\parindent=25pt \begin{minipage}[t]{.14\textwidth}
    \vspace{0pt}
    \includegraphics[width=2cm]{nerd_maddy.png} 
\end{minipage}%
\fbox{
\begin{minipage}[t]{.76\textwidth}
    \vspace{0pt}
    \textbf{Nerd Interjection!}\footnote{Image from @Ellem\_\_ on Twitter.} These sections are for me to remind you of some necessary information to solve the problems, elaborate on 
    something that I think isn't all that clear with just pure math symbols, give a helpful theorem, be an annoying piece of shit, anything, really! Just think of it as a tips and tricks section. 
\end{minipage}%
}\parindent=0pt \par I sometimes have another section called \textbf{Can we Prove it?}, where I include some interesting, not really necessary, but 
nonetheless relevant proofs.\par
\begin{minipage}[t]{.19\textwidth}
    \vspace{0pt}
    \includegraphics[width=2.8cm]{canweproveit.png} 
\end{minipage}%
\fbox{\begin{minipage}[t]{0.78\textwidth}
    \vspace{0pt} 
    \textbf{Can we Prove it?}\footnote{Thank you Mikh for this.} This is just a random proof I yoinked from our homeworks.\begin{proof} 
        ($ \implies $) Let $ x \in (A \cap B) \setminus C $. Then, $ x \in (A \cap B)$ and $ x \notin C $. \\
        \phantom{($ \implies $)} Since $x \in (A \cap B)$, $ x \in A$ and $ x \in B$. \\
        \phantom{($ \implies $)} Since $x \in A$ and $x \notin C$, $x \in (A \setminus C) $. \\
        \phantom{($ \implies $)} Since $x \in B$ and $x \notin C$, $x \in (B \setminus C) $. \\
        \phantom{($ \implies $)} Thus, $x \in (A \setminus C) \cap (B \setminus C) $. \\ 
        \\
        ($ \impliedby $) Let $ x \in (A \setminus C) \cap (B \setminus C) $. Then, $ x \in (A \setminus C) $ and $ x \in (B \setminus C) $. \\ 
        \phantom{($ \impliedby $)} Since $ x \in (A \setminus C) $, $ x \in A $ and $ x \notin C $. \\
        \phantom{($ \impliedby $)} Since $ x \in (B \setminus C) $, $ x \in B $ and $ x \notin C $. \\
        \phantom{($ \impliedby $)} Since $ x \in A $ and $ x \in B $, $ x \in (A \cap B) $. \\
        \phantom{($ \impliedby $)} Thus, $ x \in (A \cap B) \setminus C $. \\
        \\ 
        Since both sides are true, it holds that $ (A \cap B) \setminus C = (A \setminus C) \cap (B \setminus C) $. 
    \end{proof} 
\end{minipage}%
}
\par
Finally, there are blue boxes to indicate when instructions aren't obvious from the question itself, or if there are similar items that can be grouped together.\par
\parindent=25pt \transparent{0.5}
    \colorbox{CornflowerBlue}{
    \transparent{1.0}
    \begin{minipage}[c]{0.9\textwidth}
        \centering
        For items \#7 to \#12, we need to reevaluate our life decisions.
    \end{minipage}
    }\transparent{1.0}\parindent=0pt \par 
It's very important to note that this is a \textit{work in progress!} I am human, and I will make mistakes, and I cannot finish doing all the exercises within the span of one day. If you spot anything wrong, 
please feel free to message me; I will correct it as soon as possible.\par
As a final note, these are not replacements for the modules/paying attention in class, these are supplements for them. I won't explain all the topics here, and I'll assume that you at least have 
read the basics, so don't treat these reviewers as your only source of information. Our teachers spend a lot of time on the handouts, they're really good! (except when they're wrong) With that, though, I think 
I've covered all pertinent points. Good luck, and happy studying!
\pagebreak 

\section*{2.1.1: Divisibility Rules}

\end{document} 